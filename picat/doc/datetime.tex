\ignore{
\documentstyle[11pt]{report}
\textwidth 13.7cm
\textheight 21.5cm
\newcommand{\myimp}{\verb+ :- +}
\newcommand{\ignore}[1]{}
\def\definitionname{Definition}

\makeindex
\begin{document}
}

\chapter{\label{chapter:datetime}The \texttt{datetime} Module}

Picat's \texttt{datetime} module provides built-ins for retrieving the date and time. This module must be imported before use.

\begin{itemize}
\item \texttt{current\_datetime() = $DateTime$}\index{\texttt{current\_datetime/0}}: This function returns the current date and time as a structure in the form 
\begin{tabbing}
aa \= aaa \= aaa \= aaa \= aaa \= aaa \= aaa \kill
\> \texttt{date\_time($Year$, $Month$, $Day$, $Hour$, $Minute$, $Second$)}
\end{tabbing}
where the arguments are all integers, and have the following meanings and ranges.

% \begin{table}[h]
\begin{center}
\begin{tabular}{|l|l|l||}
\hline
\textbf{Argument} & \textbf{Meaning} & \textbf{Range} \\
\hline 
\hline 
$Year$ & years since 1900 & an integer \\
$Month$ & months since January & 0-11 \\
$Day$ & day of the month & 1-31 \\
$Hour$ & hours since midnight & 0-23 \\
$Minute$ & minutes after the hour & 0-59 \\
$Second$ & seconds after the minute & 0-60 \\
\hline
\end{tabular}
\end{center}
%\end{table}

\noindent
In the $Month$ argument, $0$ represents January, and $11$ represents December.  In the $Hour$ argument, $0$ represents $12$ AM, and $23$ represents $11$ PM.  In the $Second$ argument, the value $60$ represents a leap second.

\item \texttt{current\_date() = $Date$}\index{\texttt{current\_date/0}}: This function returns the current date as a structure in the form \texttt{date($Year$, $Month$, $Day$)}, where the arguments have the meanings and ranges that are defined above.

\item \texttt{current\_day() = $WDay$}\index{\texttt{current\_day/0}}: This function returns the number of days since Sunday, in the range 0 to 6. 

\item \texttt{current\_time() = $Time$}\index{\texttt{current\_time/0}}: This function returns the current time as a structure in the form \texttt{time($Hour$, $Minute$, $Second$)}, where the arguments have the meanings and ranges that are defined above.

\end{itemize}
\ignore{
\end{document}
}
